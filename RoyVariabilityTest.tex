\documentclass[12pt, a4paper]{report}
\usepackage{epsfig}
\usepackage{subfigure}
%\usepackage{amscd}
\usepackage{amssymb}
\usepackage{amsbsy}
\usepackage{amsthm}
%\usepackage[dvips]{graphicx}
\usepackage{natbib}
\bibliographystyle{chicago}
\usepackage{vmargin}
% left top textwidth textheight headheight
% headsep footheight footskip
\setmargins{3.0cm}{2.5cm}{15.5 cm}{22cm}{0.5cm}{0cm}{1cm}{1cm}
\renewcommand{\baselinestretch}{1.5}
\pagenumbering{arabic}
\theoremstyle{plain}
\newtheorem{theorem}{Theorem}[section]
\newtheorem{corollary}[theorem]{Corollary}
\newtheorem{ill}[theorem]{Example}
\newtheorem{lemma}[theorem]{Lemma}
\newtheorem{proposition}[theorem]{Proposition}
\newtheorem{conjecture}[theorem]{Conjecture}
\newtheorem{axiom}{Axiom}
\theoremstyle{definition}
\newtheorem{definition}{Definition}[section]
\newtheorem{notation}{Notation}
\theoremstyle{remark}
\newtheorem{remark}{Remark}[section]
\newtheorem{example}{Example}[section]
\renewcommand{\thenotation}{}
\renewcommand{\thetable}{\thesection.\arabic{table}}
\renewcommand{\thefigure}{\thesection.\arabic{figure}}
\title{Research notes: linear mixed effects models}
\author{ } \date{ }


\begin{document}
\author{Kevin O'Brien}
\title{Spring 2011}

\addcontentsline{toc}{section}{Bibliography}

%----------------------------------------------------------------------------------------%

\newpage
\chapter{LME models for MCS}







%-----------------------------------------------------------------------------------------------------%



%-----------------------------------------------------------------------------------------------------%
\newpage
\subsection{Roy's variability tests}
Variability tests proposed by \citet{Roy2009} affords the opportunity to expand upon Carstensen's approach.

The first test allows of the comparison the begin-subject variability of two methods. Similarly, the second test
assesses the within-subject variability of two methods. A third test is a test that compares the overall variability of the two methods.

The tests are implemented by fitting a specific LME model, and three variations thereof, to the data. These three variant models introduce equality constraints that act null hypothesis cases.

Other important aspects of the method comparison study are consequent. The limits of agreement are computed using the results of the first model.




%-----------------------------------------------------------------------------------------------------%
\bibliography{transferbib}
\end{document} 