\documentclass[12pt, a4paper]{article}
\usepackage{epsfig}
\usepackage{subfigure}
%\usepackage{amscd}
\usepackage{amssymb}
\usepackage{amsbsy}
\usepackage{amsthm}
%\usepackage[dvips]{graphicx}
\usepackage{natbib}
\bibliographystyle{chicago}
\usepackage{vmargin}
% left top textwidth textheight headheight
% headsep footheight footskip
\setmargins{3.0cm}{2.5cm}{15.5 cm}{22cm}{0.5cm}{0cm}{1cm}{1cm}
\renewcommand{\baselinestretch}{1.5}
\pagenumbering{arabic}
\theoremstyle{plain}
\newtheorem{theorem}{Theorem}[section]
\newtheorem{corollary}[theorem]{Corollary}
\newtheorem{ill}[theorem]{Example}
\newtheorem{lemma}[theorem]{Lemma}
\newtheorem{proposition}[theorem]{Proposition}
\newtheorem{conjecture}[theorem]{Conjecture}
\newtheorem{axiom}{Axiom}
\theoremstyle{definition}
\newtheorem{definition}{Definition}[section]
\newtheorem{notation}{Notation}
\theoremstyle{remark}
\newtheorem{remark}{Remark}[section]
\newtheorem{example}{Example}[section]
\renewcommand{\thenotation}{}
\renewcommand{\thetable}{\thesection.\arabic{table}}
\renewcommand{\thefigure}{\thesection.\arabic{figure}}
\title{Roy's Methodology}
\author{ } \date{ }

\begin{document}
%----------------------------------------------------------------------------------------%
\tableofcontents
\newpage
\section{Roy's LME approach}

\newpage



The methodology uses a linear mixed effects regression fit using
compound symmetry (CS) correlation structure on \textbf{V}.


$\Lambda = \frac{\mbox{max}_{H_{0}}L}{\mbox{max}_{H_{1}}L}$

\newpage

\citet{ARoy2009} considers the problem of assessing the agreement
between two methods with replicate observations in a doubly
multivariate set-up using linear mixed effects models.


\citet{ARoy2009} uses examples from \citet{BA86} to be able to
compare both types of analysis.



For the the RV-IC comparison, $\hat{D}$ is given by


\begin{equation}
\hat{D}= \left[ \begin{array}{cc}
  1.6323 & 1.1427  \\
  1.1427 & 1.4498 \\
\end{array} \right]
\end{equation}

The estimate for the within-subject variance covariance matrix is
given by
\begin{equation}
\hat{\Sigma}= \left[ \begin{array}{cc}
  0.1072 & 0.0372  \\
  0.0372 & 0.1379  \\
\end{array}\right]
\end{equation}
The estimated overall variance covariance matrix for the the 'RV
vs IC' comparison is given by
\begin{equation}
Block \Omega_{i}= \left[ \begin{array}{cc}
  1.7396 & 1.1799  \\
  1.1799 & 1.5877  \\
\end{array} \right].
\end{equation}

 The power of the
likelihood ratio test may depends on specific sample size and the
specific number of  replications, and the author proposes
simulation studies to examine this further.

\section{Roy's LME methodology for assessing agreement}

\citet{Barnhart}  describes the sources of disagreement as
differing population means, different between-subject variances,
different within-subject variances between two methods and poor
correlation between measurements of two methods.


\citet{ARoy2009}proposes the use of LME models to perform a test
on two methods of agreement to determine whether they can be used
interchangeably.

Bivariate correlation coefficients have been shown to be of
limited use in method comparison studies \citep{BA86}. However,
recently correlation analysis has been developed to cope with
repeated measurements, enhancing their potential usefulness. Roy
incorporates the use of correlation into his methodology.


\citet{ARoy2009} considers the problem of assessing the agreement
between two methods with replicate observations in a doubly
multivariate set-up using linear mixed effects models.


\citet{ARoy2009} uses examples from \citet{BA86} to be able to
compare both types of analysis.

\citet{ARoy2009} proposes a LME based approach with Kronecker
product covariance structure with doubly multivariate setup to
assess the agreement between two methods. This method is designed
such that the data may be unbalanced and with unequal numbers of
replications for each subject.

\citet{ARoy2009} considers four independent hypothesis tests.
\begin{itemize}
\item Testing of hypotheses of differences between the means of
two methods\item Testing of hypotheses in between subject
variabilities in two methods, \item Testing of hypotheses of
differences in within-subject variability of the two methods,
\item Testing of hypotheses in differences in overall variability
of the two methods.
\end{itemize}


\section{Replicates}
Measurements taken in quick succession by the same observer using the same instrument on the same subject can be considered true replicates. \citet{Roy2009} notes that some measurements may not be `true' replicates.

Roy's methodology assumes the use of `true replicates'. However data may not be collected in this way. In such cases, the correlation matrix on the replicates may require a different structure, such as the autoregressive order one $AR(1)$ structure. However determining MLEs with such a structure would be computational intense, if possible at all.

\section{Roy's methodology with single measurements}

\section{Correlation indices}
\citet{roy} remarks that PROC MIXED only gives overall correlation coefficients, but not their variances. Consequently it is not possible to carry out inferences based on all overall correlation coefficients.


\section{Roy's examples}
Roy provides three case studies, using data sets well known in method comparison studies, to demonstrate how the methodology should be used.


%--------------------------------------------------------------------------Example 1a  ----  JSR Data %

The first case study is the Systolic blood pressure data, taken from \citet{BA99}.


%--------------------------------------------------------------------------Example 1b  ----  JSR Data %

To complete the study, the relevant values are provided for the $R \mbox{vs} S$ comparison also.

%--------------------------------------------------------------------------Example 2  ----  PEFR Data %

The second data set, a comparison of two peak expiratory flow rate measurements, is referenced by \citet{BA86}.


%--------------------------------------------------------------------------Example 3 Cardiac Ejection Fraction Data %
The last case study is also based on a data set from  \citet{BA99}. It contains the measurements of left ventricular cardiac eject fraction, measured by impedance cartography and radionuclide ventriculography, on twelve patients.
The number of replicated differs for each patient.

The bias is shown to be $0.7040$, with a p-value of $0.0204$. The MLEa of the between-method and within-method variance-covariance matrices of methods $RV$ and $IC$ are given by

\begin{equation}\hat{D}=\left(
                                      \begin{array}{cc}
                                        1.6323 & 1.1427 \\
                                        1.1427 & 1.4498 \\
                                      \end{array}
                                    \right),
\end{equation}



\begin{equation}\hat{\Sigma}=\left(
                                      \begin{array}{cc}
                                        1.6323 & 1.1427 \\
                                        1.1427 & 1.4498 \\
                                      \end{array}
                                    \right).
\end{equation}

\citet{roy} notes that these are the same estimate for variance as given by \citet{BA99}.


The repeatability coefficients are determined to be $0.9080$ for the RV method and $1.0293$ for the IC method.

From the estimated $\boldsymbol{\Omega_{i}}$ correlation matrix, the overall correlation coefficient is $0.7100$.
The overall correllation coefficients between two methods RV and IC are $0.9384$ and $0.9131$ respectively.

\citet{roy} concludes that is appropriate to switch between the two methods if needed.

%--------------------------------------------------------------------------Example 4 Coronary Artery Calcium data%

\citet{haber}

\citet{roy} recommends to not switch between the two method.

\section{LME}

Fitting model according to Roy

\newpage
\begin{verbatim}
Linear mixed-effects model fit by REML
 Data: BA99
       AIC      BIC    logLik
  4319.707 4336.629 -2155.853

Random effects:
 Formula: ~1 | subj
        (Intercept) Residual
StdDev:    29.39085 12.44454

Fixed effects: ob.js ~ method
                Value Std.Error  DF  t-value p-value
(Intercept) 127.40784  3.281757 424 38.82306       0

methodS     15.61961   1.102107 424 14.17250       0

 Correlation:
        (Intr)
methodS -0.168

Standardized Within-Group Residuals:
        Min          Q1         Med          Q3         Max
-3.61292639 -0.42538402 -0.02467651  0.40166235  4.84280044

Number of Observations: 510 Number of Groups: 85

\end{verbatim}

%---------------------------------------------------------------------------------------------------%
\newpage
\bibliography{transferbib}
\end{document}
